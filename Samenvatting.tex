\documentclass[11pt, a4paper]{article}
\usepackage[latin1]{inputenc}
\usepackage[dutch]{babel}
\usepackage{csquotes}
\usepackage{amsmath}
\usepackage{amsfonts}
\usepackage{amssymb}
\usepackage[backend=biber, style=numeric, citestyle=numeric-comp, sorting = none]{biblatex}
\author{Stef Tweepenninckx}
\title{Samenvatting Logica voor Informatici 2017-2018}


%define printtitle
\makeatletter
\def\printtitle{                 
    {\large \@title}} 
\makeatother

%define printauthor
\makeatletter                       
\def\printauthor{                  
    {\large \@author}}              
\makeatother

\begin{document}
\begin{titlepage}
\newcommand{\HRule}{\rule{\linewidth}{0.5mm}} 
\center 
\textsc{\LARGE Logica voor Informatici}\\[1.5cm] 
\HRule \\[0.4cm]

{\huge \bfseries \printtitle}\\[0.4cm] 
\HRule \\[0.4cm]

\Large \emph{Authors:}\\
 \textsc{\printauthor}\\[3cm]

{\large \textsc{1 februari 2018}}\\[3cm] 

\vfill 
\end{titlepage}
\pagenumbering{roman}
\tableofcontents
\newpage
\pagenumbering{arabic}

%========================================================================================
\section{Inleiding: logica en zijn rol in informatica}
%========================================================================================
\textit{Eventueel doorlezen p 6-13}, \textbf{Goed lezen p 13-17}

%========================================================================================
\section{De predikatenlogica}
%========================================================================================
\subsection{Inleiding tot de predikatenlogica}
\textit{Doorlezen p 18-24}
\subsection{Syntax van de predikatenlogica}
Een term is een string die kan bekomen worden door herhaaldelijke toepassing van de volgende regels: objectsymbool is term, als t\textsubscript{1}, ..., t\textsubscript{n} termen zijn en G een \textit{n}-voudige functiesymbool, dan is G(t\textsubscript{1}, ..., t\textsubscript{n}) een term.\\

\subsection{Formele en informele semantiek van predikatenlogica}
\subsection{Bewijzen van waarheid in een structuur}
\subsection{Bewijzen van waarheid in een oneindige structuur}
\subsection{Construeren van modellen van een zin}
\subsection{Geo-werelden en Decawerelden}
\subsection{Pragmatiek: methodologie en probleemgevallen}

%========================================================================================
\section{Logisch gevolg en redeneren}
%========================================================================================

%========================================================================================
\section{Modelleren en redeneren in informatica-toepassingen}
%========================================================================================

%========================================================================================
\section{Algoritmes, berekenbaarheid, het Halting problem, Onbeslisbaarheid en de Onvolledigheidsstelling van G\"odel}
%========================================================================================

\end{document}