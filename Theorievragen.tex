\documentclass[11pt, a4paper]{article}
\usepackage[latin1]{inputenc}
\usepackage[dutch]{babel}
\usepackage{csquotes}
\usepackage{amsmath}
\usepackage{amsfonts}
\usepackage{amssymb}
\usepackage[backend=biber, style=numeric, citestyle=numeric-comp, sorting = none]{biblatex}
\author{Stef Tweepenninckx}
\title{Antwoorden Theorievragen 2017-2018}


%define printtitle
\makeatletter
\def\printtitle{                 
    {\large \@title}} 
\makeatother

%define printauthor
\makeatletter                       
\def\printauthor{                  
    {\large \@author}}              
\makeatother

\begin{document}
\begin{titlepage}
\newcommand{\HRule}{\rule{\linewidth}{0.5mm}} 
\center 
\textsc{\LARGE Logica voor Informatici}\\[1.5cm] 
\HRule \\[0.4cm]

{\huge \bfseries \printtitle}\\[0.4cm] 
\HRule \\[0.4cm]

\Large \emph{Authors:}\\
 \textsc{\printauthor}\\[3cm]

{\large \textsc{1 februari 2018}}\\[3cm] 

\vfill 
\end{titlepage}
\pagenumbering{roman}
\tableofcontents
\newpage
\pagenumbering{arabic}
%=======================================================================================
\section*{Belangrijke definities}
\addcontentsline{toc}{section}{Belangrijke definities}
%=======================================================================================
\subsection*{Term en formule}
\addcontentsline{toc}{subsection}{Term en formule}
\paragraph{Een term} is een string die kan bekomen worden door herhaaldelijke toepassing van de volgende regels: \begin{itemize}
\item Een objectsymbool is een term.
\item Als t\textsubscript{1}, ..., t\textsubscript{n} termen zijn en G een \textit{n}-voudige functiesymbool, dan is G(t\textsubscript{1}, ..., t\textsubscript{n}) een term.
\end{itemize}
\paragraph{Een formule} is een string die kan bekomen worden door herhaaldelijke toepassing van de volgende regels: \begin{itemize}
\item Als t\textsubscript{1}, ..., t\textsubscript{n} termen zijn en P/\textit{n} een relatiesymbool van ariteit \textit{n}, dan is P(t\textsubscript{1}, ..., t\textsubscript{n}) een formule. We noemen dit een \textbf{atoom}.
\item Als t\textsubscript{1}, t\textsubscript{2} termen zijn, dan is $t_1 = t_2$ een formule. Dit noemen we een gelijkheidsatoom.
\item Als A, B formules zijn, dan zijn $(\neg A)$, $(A\land B)$, $(A\lor B)$, $(A\implies B)$ en $(A\iff B)$ ook formules.
\item Als \textit{x} een variabele is en A een formule, dan zijn $(\exists x: A)$ en $(\forall x: A) $ ook formules.
\end{itemize}

\subsection*{Structuur}
\addcontentsline{toc}{subsection}{Structuur}
Een structuur \textit{U} bestaat uit een niet-lege verzameling D\textsubscript{\textit{U}}, het domein of universum van \textit{U}, en een toekenning van waarden $\tau^\textit{U}$ aan niet-logische symbolen $\tau$:\begin{itemize}
\item De waarde $c^\textit{U}$ voor een objectsymbool \textit{c} is een element uit het domein D\textsubscript{\textit{U}}. \textit{c} kan zowel een variabele als constante zijn.
\item De waarde $F^\textit{U}$ voor een functie-symbool F/\textit{n}: is een functie die n-tallen uit het domein op elementen van het domein afbeeldt.
\item De waarde $P^\textit{U}$ voor een predikaatsymbool P/\textit{n} is een \textit{n}-voudige relatie $P^\textit{U}$ in $D_\textit{U}$, dus $P^\textit{U} \subset  D_\textit{U}^\textit{n}$
\end{itemize}
We noemen $\tau^\textit{U}$ de waarde of interpretatie van $\tau$ in \textit{U}.

\subsection*{Formule A is waar in structuur U}
\addcontentsline{toc}{subsection}{Formule A is waar in structuur U}
\textbf{Definitie 1} in formularium.

\subsection*{Logische waarheid, consistentie, equivalentie, tegenstrijdigheid}
\addcontentsline{toc}{subsection}{Logische waarheid, consistentie, equivalentie, tegenstrijdigheid}
\paragraph{Logische waarheid}\hspace{0pt}\\
Een logische formule A is logisch waar, of is een \textbf{tautologie} als ze waar is in alle structuren \textit{U} die A interpreteren. Notatie: $\models A$
\paragraph{Logische consistentie}\hspace{0pt}\\
Een logische formule A is logisch consistent als ze waar is in minstens \'e\'en structuur.
\paragraph{Logische equivalentie}\hspace{0pt}\\
A is logisch equivalent met B indien A en B dezelfde waarheidswaarde hebben.
\paragraph{Logische tegenstrijdigheid}\hspace{0pt}\\
Een logische formule A is logisch inconsistent, tegenstrijdig of contradictorisch als er geen structuur bestaat waarin A waar is.

\subsection*{Inferentie}
\addcontentsline{toc}{subsection}{Inferentie}
Een inferentieprobleem is een probleem met een input en een gewenste output. De input bestaat uit \'e\'en of meerdere logische objecten. Hiermee bedoelen we:\begin{itemize}
\item een symbool of vocabularium
\item een waarde uit een domein
\item een structuur: een toekenning van waardes aan symbolen
\item een logische uitdrukking: een term, een verzamelingenuitdrukking, een formule of zin, een theorie.
\end{itemize}
De output bestaat uit \'e\'en of meerdere logische objecten die aan een bepaalde logische voorwaarde in termen van de invoer voldoen.

\subsection*{Een afbeelding is berekenbaar dmv een registermachine}
\addcontentsline{toc}{subsection}{Een afbeelding is berekenbaar dmv een registermachine}
Een afbeelding is berekenbaar dmv een registermachine als er een registermachine \textit{M} bestaat die bij iedere input $a_1, a_2, \cdots, a_k \in \mathbb{N}$ stopt na een eindig aantal stappen met output $f(a_1, \cdots, a_k)$.\\
We zeggen dat R semi-beslisbaar is dmv een registermachine als er een registermachine bestaat die voor elke input $(a_1, a_2, \cdots, a_n) \in \textit{R}$ eindigt en 1 antwoordt en voor elke input $(a_1, a_2, \cdots, a_n) \notin \textit{R}$ eindigt en 0 antwoordt ofwel niet eindigt.

%=======================================================================================
\section*{Oefeningen en inzichtsvragen}
\addcontentsline{toc}{section}{Oefeningen en inzichtsvragen}
%=======================================================================================
\subsection*{Is de gegeven formule logisch waar? Leg uit.}
\addcontentsline{toc}{subsection}{Is de gegeven formule logisch waar? Leg uit.}
\textit{Tegenvoorbeeld of bewijs met gevallen-onderscheid.}\\
Vb: $\forall x: \exists y: GroterDan(y,x)$
\vfill

\subsection*{Is de gegeven formule logisch consistent? Bewijs.}
\addcontentsline{toc}{subsection}{Is de gegeven formule logisch consistent? Bewijs.}
\textit{Voorbeeld geven van structuur waarin formule waar is of \textbf{??????}}\\
Vb: $\neg (\forall x: \exists y: x=y)$
\vfill
\newpage

\subsection*{Is de gegeven formule logisch equivalent met een andere formule? Leg uit en bewijs.}
\addcontentsline{toc}{subsection}{Is de gegeven formule logisch equivalent met een andere formule? Leg uit en bewijs.}
\textit{Waarheidstabellen}\\
Vb: \textbf{ander voorbeeld zie p 21}
\vfill

%=======================================================================================
\section*{Stellingen}
\addcontentsline{toc}{section}{Stellingen}
%=======================================================================================
\subsection*{Bewijzen van waarheid in een oneindige structuur}
\addcontentsline{toc}{subsection}{Bewijzen van waarheid in een oneindige structuur}

\subsection*{Generalisatiepropositie}
\addcontentsline{toc}{subsection}{Generalisatiepropositie}

\subsection*{Toevoegen van kwantoren}
\addcontentsline{toc}{subsection}{Toevoegen van kwantoren}

\subsection*{Wanneer is een propositionele zin in disjunctieve normaalvorm consistent?}
\addcontentsline{toc}{subsection}{Wanneer is een propositionele zin in disjunctieve normaalvorm consistent?}

\subsection*{Bewijs van correctheid van KE-bewijzen}
\addcontentsline{toc}{subsection}{Bewijs van correctheid van KE-bewijzen}

\subsection*{Onbeslisbaarheid van het stop-probleem}
\addcontentsline{toc}{subsection}{Onbeslisbaarheid van het stop-probleem}

\subsection*{Bewijs van het bestaan van een algoritme voor KE-bewijzen}
\addcontentsline{toc}{subsection}{Bewijs van het bestaan van een algoritme voor KE-bewijzen}





\end{document}