\documentclass[11pt, a4paper]{article}
\usepackage[latin1]{inputenc}
\usepackage[dutch]{babel}
\usepackage{csquotes}
\usepackage{amsmath}
\usepackage{amsfonts}
\usepackage{amssymb}
\usepackage[backend=biber, style=numeric, citestyle=numeric-comp, sorting = none]{biblatex}
\author{Stef Tweepenninckx}
\title{Antwoorden Theorievragen 2017-2018}


%define printtitle
\makeatletter
\def\printtitle{                 
    {\large \@title}} 
\makeatother

%define printauthor
\makeatletter                       
\def\printauthor{                  
    {\large \@author}}              
\makeatother

\begin{document}
\begin{titlepage}
\newcommand{\HRule}{\rule{\linewidth}{0.5mm}} 
\center 
\textsc{\LARGE Logica voor Informatici}\\[1.5cm] 
\HRule \\[0.4cm]

{\huge \bfseries \printtitle}\\[0.4cm] 
\HRule \\[0.4cm]

\Large \emph{Authors:}\\
 \textsc{\printauthor}\\[3cm]

{\large \textsc{1 februari 2018}}\\[3cm] 

\vfill 
\end{titlepage}
\pagenumbering{roman}
\tableofcontents
\newpage
\pagenumbering{arabic}
%=======================================================================================
\section*{Belangrijke definities}
\addcontentsline{toc}{section}{Belangrijke definities}
%=======================================================================================
\subsection*{Term en formule}
\addcontentsline{toc}{subsection}{Term en formule}
Een term is een string die kan bekomen worden door herhaaldelijke toepassing van de volgende regels: \begin{itemize}
\item Een objectsymbool is een term.
\item Als t\textsubscript{1}, ..., t\textsubscript{n} termen zijn en G een \textit{n}-voudige functiesymbool, dan is G(t\textsubscript{1}, ..., t\textsubscript{n}) een term.
\end{itemize} \hspace{0pt}\\
Een formule is een string die kan bekomen worden door herhaaldelijke toepassing van de volgende regels: \begin{itemize}
\item Als t\textsubscript{1}, ..., t\textsubscript{n} termen zijn en P/\textit{n} een relatiesymbool van ariteit \textit{n}, dan is P(t\textsubscript{1}, ..., t\textsubscript{n}) een formule. We noemen dit een \textbf{atoom}.
\item Als t\textsubscript{1}, t\textsubscript{2} termen zijn, dan is $t_1 = t_2$ een formule. Dit noemen we een gelijkheidsatoom.
\item Als A, B formules zijn, dan zijn $(\neg A)$, $(A\land B)$, $(A\lor B)$, $(A\implies B)$ en $(A\iff B)$ ook formules.
\item Als \textit{x} een variabele is en A een formule, dan zijn $(\exists x: A)$ en $(\forall x: A) $ ook formules.
\end{itemize}


\subsection*{Structuur}
\addcontentsline{toc}{subsection}{Structuur}

\subsection*{Formule A is waar in structuur U}
\addcontentsline{toc}{subsection}{Formule A is waar in structuur U}

\subsection*{Logische waarheid, consistentie, equivalentie, tegenstrijdigheid}
\addcontentsline{toc}{subsection}{Logische waarheid, consistentie, equivalentie, tegenstrijdigheid}

\subsection*{Inferentie}
\addcontentsline{toc}{subsection}{Inferentie}

\subsection*{Een afbeelding is berekenbaar dmv een registermachine}
\addcontentsline{toc}{subsection}{Een afbeelding is berekenbaar dmv een registermachine}


%=======================================================================================
\section*{Oefeningen en inzichtsvragen}
\addcontentsline{toc}{section}{Oefeningen en inzichtsvragen}
%=======================================================================================

%=======================================================================================
\section*{Stellingen}
\addcontentsline{toc}{section}{Stellingen}
%=======================================================================================

\end{document}